\documentclass{article}
% Change "article" to "report" to get rid of page number on title page
\usepackage{amsmath,amsfonts,amsthm,amssymb}
\usepackage{setspace}
\usepackage{fancyhdr}
\usepackage{lastpage}
\usepackage{extramarks}
\usepackage{chngpage}
\usepackage{soul}
\usepackage[usenames,dvipsnames]{color}
\usepackage{graphicx,float,wrapfig}
\usepackage{ifthen}
\usepackage{listings}
\usepackage{courier}
\usepackage{booktabs}

\definecolor{MyDarkGreen}{rgb}{0.0,0.4,0.0}

% For faster processing, load Matlab syntax for listings
\lstloadlanguages{Matlab}%
\lstset{language=Matlab,
        frame=single,
        basicstyle=\small\ttfamily,
        keywordstyle=[1]\color{Blue}\bf,
        keywordstyle=[2]\color{Purple},
        keywordstyle=[3]\color{Blue}\underbar,
        identifierstyle=,
        commentstyle=\usefont{T1}{pcr}{m}{sl}\color{MyDarkGreen}\small,
        stringstyle=\color{Purple},
        showstringspaces=false,
        tabsize=5,
        % Put standard MATLAB functions not included in the default
        % language here
        morekeywords={xlim,ylim,var,alpha,factorial,poissrnd,normpdf,normcdf,
        getgenbank,basecount,dimercount,seqshoworfs,cpgisland,fieldnames,getfield,
        codoncount,nt2int},
        % Put MATLAB function parameters here
        morekeywords=[2]{on, off, interp},
        % Put user defined functions here
        morekeywords=[3]{FindESS},
        morecomment=[l][\color{Blue}]{...},
        numbers=left,
        firstnumber=1,
        numberstyle=\tiny\color{Blue},
        stepnumber=5
        }

% In case you need to adjust margins:
\topmargin=-0.45in      %
\evensidemargin=0in     %
\oddsidemargin=0in      %
\textwidth=6.5in        %
\textheight=9.0in       %
\headsep=0.25in         %

% Homework Specific Information
\newcommand{\hmwkTitle}{Exercise 2}
\newcommand{\hmwkSubTitle}{T-61.5120}
\newcommand{\hmwkDueDate}{September 19, 2012}
\newcommand{\hmwkClass}{Computational Genomics}
\newcommand{\hmwkClassTime}{}
\newcommand{\hmwkClassInstructor}{Kirsti Laurila}
\newcommand{\hmwkAuthorName}{Sakari Cajanus}

% Setup the header and footer
\pagestyle{fancy}                                                       %
\lhead{\hmwkAuthorName}                                                 %
\chead{\hmwkClass\ (\hmwkClassInstructor\ \hmwkClassTime): \hmwkTitle}  %
\rhead{\firstxmark}                                                     %
\lfoot{\lastxmark}                                                      %
\cfoot{}                                                                %
\rfoot{Page\ \thepage\ of\ \protect\pageref{LastPage}}                  %
\renewcommand\headrulewidth{0.4pt}                                      %
\renewcommand\footrulewidth{0.4pt}                                      %

%%%%%%%%%%%%%%%%%%%%%%%%%%%%%%%%%%%%%%%%%%%%%%%%%%%%%%%%%%%%%
% Make title
\title{\vspace{2in}\textmd{\textbf{\hmwkClass:\ \hmwkTitle\ifthenelse{\equal{\hmwkSubTitle}{}}{}{\\\large\hmwkSubTitle}}}\\\normalsize\vspace{0.1in}\small{Due\ on\ \hmwkDueDate}\\\vspace{0.1in}\large{\textit{\hmwkClassInstructor\ \hmwkClassTime}}\vspace{3in}}
\date{}
\author{\textbf{\hmwkAuthorName}}
%%%%%%%%%%%%%%%%%%%%%%%%%%%%%%%%%%%%%%%%%%%%%%%%%%%%%%%%%%%%%

% This is used to trace down (pin point) problems
% in latexing a document:
%\tracingall

%%%%%%%%%%%%%%%%%%%%%%%%%%%%%%%%%%%%%%%%%%%%%%%%%%%%%%%%%%%%%
% Some tools

% Use these to play with the headers and footers
\newcommand{\enterProblemHeader}[1]{\nobreak\extramarks{#1}{#1 continued on next page\ldots}\nobreak%
                                    \nobreak\extramarks{#1 (continued)}{#1 continued on next page\ldots}\nobreak}%
\newcommand{\exitProblemHeader}[1]{\nobreak\extramarks{#1 (continued)}{#1 continued on next page\ldots}\nobreak%
                                   \nobreak\extramarks{#1}{}\nobreak}%

\setcounter{secnumdepth}{0}
\newcommand{\homeworkProblem}[3]
 {\enterProblemHeader{#1}
   \ifthenelse{\equal{#2}{}}%
        {\subsection{#1}}%
        {\subsection{#1 (#2)}}%
               #3
   \exitProblemHeader{#1}}

%%% I think \captionwidth (commented out below) can go away
%%%
%% Edits the caption width
%\newcommand{\captionwidth}[1]{%
%  \dimen0=\columnwidth   \advance\dimen0 by-#1\relax
%  \divide\dimen0 by2
%  \advance\leftskip by\dimen0
%  \advance\rightskip by\dimen0
%}

% Includes a figure
% The first parameter is the label, which is also the name of the figure
%   with or without the extension (e.g., .eps, .fig, .png, .gif, etc.)
%   IF NO EXTENSION IS GIVEN, LaTeX will look for the most appropriate one.
%   This means that if a DVI (or PS) is being produced, it will look for
%   an eps. If a PDF is being produced, it will look for nearly anything
%   else (gif, jpg, png, et cetera). Because of this, when I generate figures
%   I typically generate an eps and a png to allow me the most flexibility
%   when rendering my document.
% The second parameter is the width of the figure normalized to column width
%   (e.g. 0.5 for half a column, 0.75 for 75% of the column)
% The third parameter is the caption.
\newcommand{\scalefig}[3]{
  \begin{figure}[ht!]
    % Requires \usepackage{graphicx}
    \centering
    \includegraphics[width=#2\columnwidth]{#1}
    %%% I think \captionwidth (see above) can go away as long as
    %%% \centering is above
    %\captionwidth{#2\columnwidth}%
    \caption{#3}
    \label{#1}
  \end{figure}}

% Includes a MATLAB script.
% The first parameter is the label, which also is the name of the script
%   without the .m.
% The second parameter is the optional caption.
\newcommand{\matlabscript}[2]
  {\begin{itemize}\item[]\lstinputlisting[caption=#2,label=#1]{#1.m}\end{itemize}}

%%%%%%%%%%%%%%%%%%%%%%%%%%%%%%%%%%%%%%%%%%%%%%%%%%%%%%%%%%%%%

%%%%%%%%%%%%%%%%%%%%%%%%%%%%%%%%%%%%%%%%%%%%%%%%%%%%%%%%%%%%%
%%%%%%%%%% The main document content
%%%%%%%%%%%%%%%%%%%%%%%%%%%%%%%%%%%%%%%%%%%%%%%%%%%%%%%%%%%%%

\begin{document}
\begin{spacing}{1.1}
\maketitle
% Uncomment the \tableofcontents line to get a Contents on front page
%\tableofcontents
\newpage

\homeworkProblem{Problem 1}{Local alignment}%
{
    Raw score is determined as the sum of the substitution matrix values of the
    sequences
        \begin{align}
    S=\sum_{i=1}^L F_{i,j}
    \end{align}
    and using the BLOSUM62 as the scoring matrix, we get the raw score value of
        \begin{align}
    S=\sum_{i=1}^L F_{i,j}=10.
        \end{align}
        \emph{NOTE: It seems that the swalign-function return sum of blosum-matrix values divided by 2, so this is what I did here, too. See section Code.}

    Then, to find out the bitscore (``a normalized score expressed in bits that
    lets you estimate the magnitude of the search space you would have to look
    through before you would expect to find an score as good as or better than
    this one by chance.''\cite{hc}) we need to calculate
        \begin{align}
    S'=\frac{\lambda S -\ln(K)}{\ln(2)}=\frac{0.31\times10 -
    \ln(0.038)}{\ln(2)}=9.19
        \end{align}
    The expectation value can be calculated with
        \begin{align}
    E=Kmn\mathrm{e}^{-\lambda S}=0.038\times 5134782 \times 30 \times \mathrm{e}^{-0.31*10}=2.637\times10^{5}
        \end{align}
    Based on these values, it can be said that the alignment is not significant.
 } % End homework solution
 \clearpage
\homeworkProblem{Problem 2}{Amino Acids}%
{
    Positively charged amino acids:
    \begin{itemize}
        \item Arginine (R)
        \item Lysine (K)
    \end{itemize}
    Negatively charged amino acids:
    \begin{itemize}
        \item Glutamate (G)
        \item Aspartate (A)
    \end{itemize}

\begin{table}[ht]
\caption{BLOSUM 62 scores for negatively and positively charged amino acids\vspace{2pt}}
\centering
\begin{tabular}{lrrrr}
\toprule
   & R & K & G & A \\
\hline
 R &  5 &  2 & -2 & -1  \\
 K &  2 &  5 & -2 & -1  \\
 G & -2 & -2 &  6 &  0  \\
 R & -1 & -1 &  0 &  4  \\
\bottomrule
\end{tabular}
\end{table}

\begin{table}[ht]
\caption{BLOSUM 62 scores for charged \& hydrophobic aas\vspace{2pt}}
\centering
\begin{tabular}{lrrrr}
\toprule
   & V & I & L & M \\
\hline
 R & -3 & -3 & -2 & -1  \\
 K & -2 & -3 & -2 & -1  \\
 G & -3 & -4 & -4 & -3  \\
 R &  0 & -1 & -1 & -1  \\
\bottomrule
\end{tabular}
\end{table}

} % End homework solution
 \clearpage
\homeworkProblem{Problem 4}{Smith-Waterman algorithm}%
{
The code for the Smith-Waterman algorithm is shown below. It did produce similar results as the default swalign-function.
\matlabscript{ex2_pr4}{}

 } % End homework solution
\clearpage
\appendix
\section{Code}
\matlabscript{ex2}{}
\end{spacing}

\begin{thebibliography}{9}
\bibitem{hc}
http://homepages.ulb.ac.be/~dgonze/TEACHING/stat\_scores.pdf
   retrieved Sept. 25 2012.
\end{thebibliography}
\end{document}

%%%%%%%%%%%%%%%%%%%%%%%%%%%%%%%%%%%%%%%%%%%%%%%%%%%%%%%%%%%%%

%----------------------------------------------------------------------%
% The following is copyright and licensing information for
% redistribution of this LaTeX source code; it also includes a liability
% statement. If this source code is not being redistributed to others,
% it may be omitted. It has no effect on the function of the above code.
%----------------------------------------------------------------------%
% Copyright (c) 2007, 2008, 2009, 2010, 2011 by Theodore P. Pavlic
%
% Unless otherwise expressly stated, this work is licensed under the
% Creative Commons Attribution-Noncommercial 3.0 United States License. To
% view a copy of this license, visit
% http://creativecommons.org/licenses/by-nc/3.0/us/ or send a letter to
% Creative Commons, 171 Second Street, Suite 300, San Francisco,
% California, 94105, USA.
%
% THE SOFTWARE IS PROVIDED "AS IS", WITHOUT WARRANTY OF ANY KIND, EXPRESS
% OR IMPLIED, INCLUDING BUT NOT LIMITED TO THE WARRANTIES OF
% MERCHANTABILITY, FITNESS FOR A PARTICULAR PURPOSE AND NONINFRINGEMENT.
% IN NO EVENT SHALL THE AUTHORS OR COPYRIGHT HOLDERS BE LIABLE FOR ANY
% CLAIM, DAMAGES OR OTHER LIABILITY, WHETHER IN AN ACTION OF CONTRACT,
% TORT OR OTHERWISE, ARISING FROM, OUT OF OR IN CONNECTION WITH THE
% SOFTWARE OR THE USE OR OTHER DEALINGS IN THE SOFTWARE.
%----------------------------------------------------------------------%
