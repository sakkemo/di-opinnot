\documentclass{article}
% Change "article" to "report" to get rid of page number on title page
\usepackage{amsmath,amsfonts,amsthm,amssymb}
\usepackage{setspace}
\usepackage{fancyhdr}
\usepackage{lastpage}
\usepackage{extramarks}
\usepackage{chngpage}
\usepackage{soul}
\usepackage[usenames,dvipsnames]{color}
\usepackage{graphicx,float,wrapfig}
\usepackage{ifthen}
\usepackage{listings}
\usepackage{courier}
\usepackage{booktabs}

\definecolor{MyDarkGreen}{rgb}{0.0,0.4,0.0}

% For faster processing, load Matlab syntax for listings
\lstloadlanguages{Matlab}%
\lstset{language=Matlab,
        frame=single,
        basicstyle=\small\ttfamily,
        keywordstyle=[1]\color{Blue}\bf,
        keywordstyle=[2]\color{Purple},
        keywordstyle=[3]\color{Blue}\underbar,
        identifierstyle=,
        commentstyle=\usefont{T1}{pcr}{m}{sl}\color{MyDarkGreen}\small,
        stringstyle=\color{Purple},
        showstringspaces=false,
        tabsize=5,
        % Put standard MATLAB functions not included in the default
        % language here
        morekeywords={xlim,ylim,var,alpha,factorial,poissrnd,normpdf,normcdf,
        getgenbank,basecount,dimercount,seqshoworfs,cpgisland,fieldnames,getfield,
        codoncount,nt2int},
        % Put MATLAB function parameters here
        morekeywords=[2]{on, off, interp},
        % Put user defined functions here
        morekeywords=[3]{FindESS},
        morecomment=[l][\color{Blue}]{...},
        numbers=left,
        firstnumber=1,
        numberstyle=\tiny\color{Blue},
        stepnumber=5
        }

% In case you need to adjust margins:
\topmargin=-0.45in      %
\evensidemargin=0in     %
\oddsidemargin=0in      %
\textwidth=6.5in        %
\textheight=9.0in       %
\headsep=0.25in         %

% Homework Specific Information
\newcommand{\hmwkTitle}{Exercise 1}
\newcommand{\hmwkSubTitle}{T-61.5120}
\newcommand{\hmwkDueDate}{September 19, 2012}
\newcommand{\hmwkClass}{Computational Genomics}
\newcommand{\hmwkClassTime}{}
\newcommand{\hmwkClassInstructor}{Kirsti Laurila}
\newcommand{\hmwkAuthorName}{Sakari Cajanus}

% Setup the header and footer
\pagestyle{fancy}                                                       %
\lhead{\hmwkAuthorName}                                                 %
\chead{\hmwkClass\ (\hmwkClassInstructor\ \hmwkClassTime): \hmwkTitle}  %
\rhead{\firstxmark}                                                     %
\lfoot{\lastxmark}                                                      %
\cfoot{}                                                                %
\rfoot{Page\ \thepage\ of\ \protect\pageref{LastPage}}                  %
\renewcommand\headrulewidth{0.4pt}                                      %
\renewcommand\footrulewidth{0.4pt}                                      %

%%%%%%%%%%%%%%%%%%%%%%%%%%%%%%%%%%%%%%%%%%%%%%%%%%%%%%%%%%%%%
% Make title
\title{\vspace{2in}\textmd{\textbf{\hmwkClass:\ \hmwkTitle\ifthenelse{\equal{\hmwkSubTitle}{}}{}{\\\large\hmwkSubTitle}}}\\\normalsize\vspace{0.1in}\small{Due\ on\ \hmwkDueDate}\\\vspace{0.1in}\large{\textit{\hmwkClassInstructor\ \hmwkClassTime}}\vspace{3in}}
\date{}
\author{\textbf{\hmwkAuthorName}}
%%%%%%%%%%%%%%%%%%%%%%%%%%%%%%%%%%%%%%%%%%%%%%%%%%%%%%%%%%%%%

% This is used to trace down (pin point) problems
% in latexing a document:
%\tracingall

%%%%%%%%%%%%%%%%%%%%%%%%%%%%%%%%%%%%%%%%%%%%%%%%%%%%%%%%%%%%%
% Some tools

% Use these to play with the headers and footers
\newcommand{\enterProblemHeader}[1]{\nobreak\extramarks{#1}{#1 continued on next page\ldots}\nobreak%
                                    \nobreak\extramarks{#1 (continued)}{#1 continued on next page\ldots}\nobreak}%
\newcommand{\exitProblemHeader}[1]{\nobreak\extramarks{#1 (continued)}{#1 continued on next page\ldots}\nobreak%
                                   \nobreak\extramarks{#1}{}\nobreak}%

\setcounter{secnumdepth}{0}
\newcommand{\homeworkProblem}[3]
 {\enterProblemHeader{#1}
   \ifthenelse{\equal{#2}{}}%
        {\subsection{#1}}%
        {\subsection{#1 (#2)}}%
               #3
   \exitProblemHeader{#1}}

%%% I think \captionwidth (commented out below) can go away
%%%
%% Edits the caption width
%\newcommand{\captionwidth}[1]{%
%  \dimen0=\columnwidth   \advance\dimen0 by-#1\relax
%  \divide\dimen0 by2
%  \advance\leftskip by\dimen0
%  \advance\rightskip by\dimen0
%}

% Includes a figure
% The first parameter is the label, which is also the name of the figure
%   with or without the extension (e.g., .eps, .fig, .png, .gif, etc.)
%   IF NO EXTENSION IS GIVEN, LaTeX will look for the most appropriate one.
%   This means that if a DVI (or PS) is being produced, it will look for
%   an eps. If a PDF is being produced, it will look for nearly anything
%   else (gif, jpg, png, et cetera). Because of this, when I generate figures
%   I typically generate an eps and a png to allow me the most flexibility
%   when rendering my document.
% The second parameter is the width of the figure normalized to column width
%   (e.g. 0.5 for half a column, 0.75 for 75% of the column)
% The third parameter is the caption.
\newcommand{\scalefig}[3]{
  \begin{figure}[ht!]
    % Requires \usepackage{graphicx}
    \centering
    \includegraphics[width=#2\columnwidth]{#1}
    %%% I think \captionwidth (see above) can go away as long as
    %%% \centering is above
    %\captionwidth{#2\columnwidth}%
    \caption{#3}
    \label{#1}
  \end{figure}}

% Includes a MATLAB script.
% The first parameter is the label, which also is the name of the script
%   without the .m.
% The second parameter is the optional caption.
\newcommand{\matlabscript}[2]
  {\begin{itemize}\item[]\lstinputlisting[caption=#2,label=#1]{#1.m}\end{itemize}}

%%%%%%%%%%%%%%%%%%%%%%%%%%%%%%%%%%%%%%%%%%%%%%%%%%%%%%%%%%%%%

%%%%%%%%%%%%%%%%%%%%%%%%%%%%%%%%%%%%%%%%%%%%%%%%%%%%%%%%%%%%%
%%%%%%%%%% The main document content
%%%%%%%%%%%%%%%%%%%%%%%%%%%%%%%%%%%%%%%%%%%%%%%%%%%%%%%%%%%%%

\begin{document}
\begin{spacing}{1.1}
\maketitle
% Uncomment the \tableofcontents line to get a Contents on front page
%\tableofcontents
\newpage

\homeworkProblem{Problem 1}{Human chromosome 22/NT\_011526}%
 { % Start homework solution
 Getting the piece of the chromosome was simple using the \texttt{getgenbank}
 -function. Just taking the length of the sequence as a string, I got the length of the sequence which was 829789 nucleotides. Assuming the length of 51304566 of the whole chromosome 22\cite{hc}, this piece covers approximately $1.62 \%$ of the chromosome.

 The most abundant nucleotide in the sequence was glycine (appears 227658 times), and the most abundant dinucleotide GG (74198). The GC-content,
\begin{align}
\textrm{GC-content}=
\cfrac{G+C}{A+T+G+C}\times\ 100 \%,
\end{align}
was 54.2 \%.
 } % End homework solution
\homeworkProblem{Problem 2}{Open Reading Frames}{
Checking for Open Reading Frames of length longer than 150 nucleotides, I counted 9 ORFs in the original strand and 6 ORFs in the reverse complement. In total, there were then 15 Open Reading Frames of the required length or longer. Functions from the Bioinformatics Toolbox were used to calculate these.
}
\homeworkProblem{Problem 3}{CpG-islands}{
Using the \texttt{cpgisland}-function in the Bioinformatics Toolbox, I found in total 71 CpG-islands of length 200 or longer. The longest (13th) of these, was 1384 nucleotides. It started at nucleotide index of 200565, and ended at the index 201948.
}
\homeworkProblem{Problem 4}{2nd Order Markov Model}{
For the initial distribution, I used the ratios of dimercounts in the sequence, which resulted in the following:
\begin{table}[ht]
\caption{Initial distribution\vspace{2pt}}
\centering
\begin{tabular}{lr}
\toprule
dimer & probability\\
\hline
 AA &  0.0566\\
 AC &  0.0494\\
 AG &  0.0763\\
 AT &  0.0421\\
 CA &  0.0758\\
 CC &  0.0891\\
 CG &  0.0287\\
 CT &  0.0745\\
 GA &  0.0592\\
 GC &  0.0706\\
 GG &  0.0894\\
 GT &  0.0551\\
 TA &  0.0328\\
 TC &  0.0590\\
 TG &  0.0799\\
 TT &  0.0615\\
\bottomrule
\end{tabular}
\end{table}

For the transition matrix, I used a 2D-matrix since I didn't know how to use 3D matrices in MATLAB. This was doable with some clever indexing of the test sequence. The transition matrix that was used was the following:
\begin{displaymath}
 \bordermatrix{
 & A & C & G & T \cr
 AA & 0.3804 &    0.1740 &    0.2513 &    0.1943 \cr
 AC & 0.3207 &    0.3152 &    0.1089 &    0.2552 \cr
 AG & 0.2380 &    0.2491 &    0.3379 &    0.1751 \cr
 AT & 0.1825 &    0.2418 &    0.3013 &    0.2744 \cr
 CA & 0.1684 &    0.2679 &    0.3936 &    0.1701 \cr
 CC & 0.2769 &    0.3297 &    0.1114 &    0.2820 \cr
 CG & 0.1634 &    0.2877 &    0.3450 &    0.2040 \cr
 CT & 0.1042 &    0.2984 &    0.3984 &    0.1990 \cr
 GA & 0.2245 &    0.2110 &    0.3930 &    0.1716 \cr
 GC & 0.2594 &    0.3539 &    0.1177 &    0.2689 \cr
 GG & 0.2242 &    0.2730 &    0.3186 &    0.1842 \cr
 GT & 0.1393 &    0.2358 &    0.4171 &    0.2078 \cr
 TA & 0.2852 &    0.2150 &    0.2561 &    0.2437 \cr
 TC & 0.2766 &    0.3371 &    0.0746 &    0.3117 \cr
 TG & 0.2141 &    0.2248 &    0.3195 &    0.2416 \cr
 TT & 0.1585 &    0.2193 &    0.2394 &    0.3828 }
\end{displaymath}
Based on this model, the log-probability for the sequence 'GATACC' was -8.98.

Using the \texttt{seqwordcount}-function in the Bioinformatics Toolbox we find that the total number of occurrences of the GATA-binding site sequence 'GATACC' is 47. Comparing that to the possible amount of words, we get the relative frequency $5.6641\times10^{-05}$.
}
\appendix
\section{Code}
\matlabscript{ex1}{}
\end{spacing}

\begin{thebibliography}{9}
\bibitem{hc}
   NCBI Map Viewer: Homo Sapiens(human),
   \emph{Chromosome 22}.\\
   http://www.ncbi.nlm.nih.gov/projects/mapview/maps.cgi?taxid=9607\&chr=22\\
   retrieved Sept. 18 2012.

\end{thebibliography}
\end{document}

%%%%%%%%%%%%%%%%%%%%%%%%%%%%%%%%%%%%%%%%%%%%%%%%%%%%%%%%%%%%%

%----------------------------------------------------------------------%
% The following is copyright and licensing information for
% redistribution of this LaTeX source code; it also includes a liability
% statement. If this source code is not being redistributed to others,
% it may be omitted. It has no effect on the function of the above code.
%----------------------------------------------------------------------%
% Copyright (c) 2007, 2008, 2009, 2010, 2011 by Theodore P. Pavlic
%
% Unless otherwise expressly stated, this work is licensed under the
% Creative Commons Attribution-Noncommercial 3.0 United States License. To
% view a copy of this license, visit
% http://creativecommons.org/licenses/by-nc/3.0/us/ or send a letter to
% Creative Commons, 171 Second Street, Suite 300, San Francisco,
% California, 94105, USA.
%
% THE SOFTWARE IS PROVIDED "AS IS", WITHOUT WARRANTY OF ANY KIND, EXPRESS
% OR IMPLIED, INCLUDING BUT NOT LIMITED TO THE WARRANTIES OF
% MERCHANTABILITY, FITNESS FOR A PARTICULAR PURPOSE AND NONINFRINGEMENT.
% IN NO EVENT SHALL THE AUTHORS OR COPYRIGHT HOLDERS BE LIABLE FOR ANY
% CLAIM, DAMAGES OR OTHER LIABILITY, WHETHER IN AN ACTION OF CONTRACT,
% TORT OR OTHERWISE, ARISING FROM, OUT OF OR IN CONNECTION WITH THE
% SOFTWARE OR THE USE OR OTHER DEALINGS IN THE SOFTWARE.
%----------------------------------------------------------------------%
