\documentclass[a4paper]{article}
\usepackage[english]{babel}
\usepackage{ucs}
\usepackage[utf8x]{inputenc}
\usepackage[T1]{fontenc}
\usepackage{times}
\usepackage[pdftex]{graphicx}
\usepackage{color}
\usepackage[pdftex,colorlinks=true,citecolor=black,
            pagecolor=black,linkcolor=black,menucolor=black,
            urlcolor=black]{hyperref}
\usepackage{eufrak}
\usepackage{amsmath}
\usepackage{amsbsy}
\usepackage{eucal}
\usepackage{subfigure}
\usepackage{longtable}
\usepackage{url}
\urlstyle{same}

\usepackage{natbib}

\pdfinfo{            
          /Title      (BECS-114.2601 Bayesian Modeling)
          /Author     ()
          /Keywords   ()          
}


\title{BECS-114.2601 Bayesian Modeling - Exercise set 1 (exercises 2.9 and 2.17)}
\author{Sakari Cajanus \\ 
       {\it sakari.cajanus@aalto.fi}}


\begin{document}
%--- KANSILEHTI -------------------------------------------------------------
\maketitle


\newpage
%------------------------------------------------------------------------------

\section*{Exercise 2.9}
The model used in the exercise 2.9. was Binomial observation model using
Beta-prior. The Binomial likelihood is given by
\begin{align*}
    p(y | n, \theta)= {n\choose k}\theta^y(1-\theta)^{n-y}.
\end{align*}

We were asked to determine the prior Beta-distribution
parameters $\alpha$ and $\beta$, given the mean and standard deviation of the
distribution. This can be done simply by solving the equations that give mean
and standard deviation (or variance), given the parameters
\begin{align*}
    \mu &= \frac{\alpha}{\alpha+\beta}\!  \\
    \sigma^2 &= \frac{\alpha\beta}{(\alpha+\beta)^2(\alpha+\beta+1)}\!
\end{align*}
which gives the following equations:
\begin{align*}
\alpha &= \frac{(1-\mu)\mu^2}{\sigma^2}-\mu\\
\beta &= \alpha(\frac{1}{\mu}-1)
\end{align*}
Using parameter values determined by mean 0.6 and standard deviation of 0.3
($\alpha=1,\;\beta=\frac{2}{3}$) we get a prior distribution shown in figure
\ref{fig:prior}.
\begin{figure}[ht]
  \begin{center}
      \includegraphics[]{fig_ex29prior}
    \caption{Prior distribution Beta(1,2/3)}\label{fig:prior}   
 \end{center}
\end{figure}
If we then take a random sample of 1000 Californians, of which 65 \% support
the death penalty, using the Beta-distribution prior for Binomial observation
model we can get the the posterior (Beta) distribution for the parameter
$\theta$ easily, as the Beta-distribution is a natural conjugate prior for the
Binomial model. The updating rules for for the parameters are
\begin{align*}
    \alpha &= \alpha_0 + y = 1 + 650 = 651\\
    \beta &= \beta_0 + n -y, = \frac{2}{3} +1000 -650 = 350\frac{2}{3}
\end{align*}
where $n$ is our total number of observations (1000) and $y$ is the frequency
of the death penalty supporters in our sample (650). Thus the resulting
posterior (shown in figure \ref{fig:post1}) is quite peaked at around
$\theta=0.65$ as our data swamps the prior as can be seen from the parameter
update equations.
\begin{figure}[ht]
  \begin{center}
      \includegraphics[]{fig_ex29posterior}
    \caption{Prior and posterior using the Beta(1,2/3) prior.}\label{fig:post1}   
 \end{center}
\end{figure}
Testing the effect of the prior was done using an uniform or non-informative
prior (Beta(1,1)), as well as highly informative prior assuming that the
expected value of $\theta$ is 0.5, which was Beta(1000,1000). The priors and
their effect on the posteriors are shown in figure \ref{fig:priors2}.
\begin{figure}[ht]
  \begin{center}
    \subfigure[Exercise 2.9 prior]{
      \label{fig_prior2}
      \includegraphics[]{fig_ex29prior2}
    }
    ~
    \subfigure[Exercise 2.9 posterior]{
      \label{fig_posterior2}
      \includegraphics[]{fig_ex29posterior2}
    }
    \caption{    %Here you can add the caption text for the figure.
      In figures \ref{fig_prior2} and \ref{fig_posterior2}, the effect of
non-informative prior vs. highly informative prior with mean=0.5 is
shown.}\label{fig:priors2}   
 \end{center}
\end{figure}
\clearpage
\section*{Exercise 2.17}
(a) In the first part of the exercise, we were asked to show that the
given Beta(100,100) distribution, the probability of parameter $\theta$
following this distribution being between values of 0.4 and 0.6 is greater
than 95 \%. We were also asked to show that given those assumptions we are
ambivalent as to whether the value is greater or less than 0.5. These can be
easily shown using the CDF for the Beta distribution:
\begin{align*}
    \mathrm{CDF} &=
\int_x^y\frac{x^{\alpha-1}(1-x)^{\beta-1}}{\mathrm{B}(\alpha,\beta)}\\
p(0.4<\theta<0.6) &= 
\int_{0.4}^{0.6}\frac{x^{\alpha-1}(1-x)^{\beta-1}}{\mathrm{B}(\alpha,\beta)}\approx0.9957\\
p(0\leq\theta<0.5) &= 
\int_{0}^{0.5}\frac{x^{\alpha-1}(1-x)^{\beta-1}}{\mathrm{B}(\alpha,\beta)}=0.5\\
&= p(0.5<\theta\leq1).
\end{align*}
The Beta function $\mathrm{B}(\alpha,\beta)$ is the normalization constant
and $\alpha=\beta=100$.

(b) Also in this exercise we assume similar likelihood model as we used in
the prior exercise 2.9. Thus the posterior distribution can be obtained using
similar equations,
\begin{align*}
    \alpha &= \alpha_0 + y = 100 + 1000 - 511 = 589\\
    \beta &= \beta_0 + n -y, = 100 + 511 = 611
\end{align*}
The prior and posterior distributions are shown in figure \ref{fig:post3}.
The posterior probability for $\theta>0.5$ can be calculated from the
posterior (beta(589,611)) distribution
\begin{align*}
p(0.5<\theta<1) &= 
\int_{0.5}^{1}\frac{x^{\alpha-1}(1-x)^{\beta-1}}{\mathrm{B}(\alpha,\beta)}\bigg|_{\alpha=589,\beta=611}
\approx0.26\\
\end{align*}

\begin{figure}[ht]
  \begin{center}
      \includegraphics[]{fig_ex29posterior3}
    \caption{Prior and posterior using the Beta(1,2/3) prior.}\label{fig:post3}   
 \end{center}
\end{figure}

We
were also asked to non-informative and highly informative priors, for which
the priors and posteriors are shown in figure \ref{fig:priors3}. The
non-informative prior was Beta(1,1) and the highly informative was
beta(1000,1000). As can be seen, using the highly informative prior results
in posterior distribution with mean closer to the prior mean value of 0.5,
while the posterior mean using the non-informative prior is closer to the
ratio calculated from the data (0.489).

\begin{figure}[ht]
  \begin{center}
    \subfigure[Exercise 2.17 priors]{
      \label{fig_prior3}
      \includegraphics[]{fig_ex29prior3}
    }
    ~
    \subfigure[Exercise 2.17 posteriors]{
      \label{fig_posterior3}
      \includegraphics[]{fig_ex29posterior3}
    }
    \caption{    %Here you can add the caption text for the figure.
      In figures \ref{fig_prior3} and \ref{fig_posterior3}, the effect of
non-informative prior vs. highly informative prior with mean=0.5 is
shown.}\label{fig:priors3}   
 \end{center}
\end{figure}
%\begin{figure}
%  \begin{center}
%    \subfigure[Exercise 2.9 prior]{
%      \label{fig_prior}
%      \includegraphics[]{fig_ex29prior}
%    }
%    ~
%    \subfigure[Exercise 2.9 posterior]{
%      \label{fig_posterior}
%      \includegraphics[]{fig_ex29posterior}
%    }
%    \caption{    %Here you can add the caption text for the figure.
%      In figures \ref{fig_prior} and \ref{fig_posterior}, the prior and posterior 
%      distributions for exercise 2.9. are presented.}\label{fig_example}   
% \end{center}
%\end{figure}

\end{document}

%%% Local Variables: 
%%% mode: latex
%%% TeX-master: t
%%% End: 
