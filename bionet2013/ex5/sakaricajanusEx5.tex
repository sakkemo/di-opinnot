\documentclass[a4paper]{article}
\usepackage[english]{babel}
\usepackage{ucs}
\usepackage[utf8x]{inputenc}
\usepackage[T1]{fontenc}
\usepackage{times}
\usepackage[pdftex]{graphicx}
\usepackage{color}
\usepackage[pdftex,colorlinks=true,citecolor=black,
            pagecolor=black,linkcolor=black,menucolor=black,
            urlcolor=black]{hyperref}
\usepackage{eufrak}
\usepackage{amsmath}
\usepackage{amsbsy}
\usepackage{eucal}
\usepackage{subfigure}
\usepackage{longtable}
\usepackage{url}
\usepackage{enumerate}
\urlstyle{same}

\usepackage{natbib}

\pdfinfo{
          /Title      (T-61.5110 Modeling Biological Networks)
          /Author     ()
          /Keywords   ()
}


\title{T-61.5110 Modeling Biological Networks - Problem set 5}
\author{Sakari Cajanus\\ 82036R \\
       {\it sakari.cajanus@aalto.fi}}


\begin{document}
%--- KANSILEHTI -------------------------------------------------------------
\maketitle


\newpage
%------------------------------------------------------------------------------

\section*{Problem 1}
\begin{enumerate}[a.]
    \item Protein dimerization/dissociation dynamics. is clear that the
        concentration converges. The change/curve is smooth because the
        modelling is deterministic. Figure~\ref{fig:dimer}.
        \begin{figure}[ht!]
    \begin{center}
        \includegraphics[width=0.7\textwidth]{dimer}
        \caption{Deterministic dimerization/dissociation dynamics
        (concentrations of the species).}
        \label{fig:dimer}
    \end{center}
\end{figure}
\item Stochastic kinetics. You can see that the realizations differ and that
    there are still changes in the concentration when the average
    concentrations have converged. Figure~\ref{fig:stoch}.
\begin{figure}[htp]
  \centering
  \subfigure{\label{fig:1}\includegraphics[width=0.4\textwidth]{dimer-stoch1}}
  \subfigure{\label{fig:2}\includegraphics[width=0.4\textwidth]{dimer-stoch2}}
  %\\
  %\subfigure{\label{fig:3}\includegraphics[width=0.4\textwidth]{dimer-stoch3}}
  %\subfigure{\label{fig:3}\includegraphics[width=0.4\textwidth]{dimer-stoch4}}
  \caption{Examples of stochastic kinetics for the same process.}\label{fig:stoch}
\end{figure}
\item Here are the results of running 100 simulations of the stochastic model
    for $T=20$. In figure~\ref{fig:sim1} are the concentrations of the dimer
    for 100 realizations, in figure~\ref{fig:sim2} the mean and confidence
    intervals of 3 SD and in figure~\ref{fig:sim3} a histogram of number of
    $P$ molecules at T=20 with red line showing the corresponding
    concentration in the deterministic model.
    \begin{figure}[htp]
  \centering

  \subfigure[Number of $P$ molecules with 100 simulations.]{\label{fig:sim1}\includegraphics[width=0.4\textwidth]{samples}}
  \subfigure[The mean number of molecules with 100 simulations.]{\label{fig:sim2}\includegraphics[width=0.4\textwidth]{samplemean}}
  \\
  \subfigure[Number of molecules at time T=20.]{\label{fig:sim3}\includegraphics[width=0.6\textwidth]{hist}}
  \caption{The plots for the Problem 1c.}\label{fig:sim}
  
\end{figure}
\end{enumerate}
\clearpage

\section*{Problem 2}
In the problem 2, we repeated the simulations in 1c with uncertainty in the
rate constant $k_2$. In the first simulations, $k$ was 0.2 and in the latter
$k\sim U(0.1,0.4)$. The mean of this uniform distribution is greater than 0.2,
which shifts the balance such that the concentration of $P$ is greater. The
results are shown in figure~\ref{fig:sim-2}.
\begin{figure}[htp]
  \centering

  \subfigure[Number of $P$ molecules with 100 simulations.]{\label{fig:sim-2-1}\includegraphics[width=0.4\textwidth]{samples2}}
  \subfigure[The mean of number of molecules with 100 simulations.]{\label{fig:sim-2-2}\includegraphics[width=0.4\textwidth]{samplemean2}}
  \\
  \subfigure[Number of molecules at time T=20.]{\label{fig:sim-2-3}\includegraphics[width=0.6\textwidth]{hist2}}
  \caption{The plots for the Problem 2.}\label{fig:sim-2}
  
\end{figure}
\end{document}
%%% Local Variables:
%%% mode: latex
%%% TeX-master: t
%%% End:
