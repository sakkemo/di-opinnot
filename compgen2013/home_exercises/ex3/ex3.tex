\documentclass{article}
% Change "article" to "report" to get rid of page number on title page
\usepackage{fontenc}
\usepackage{amsmath,amsfonts,amsthm,amssymb}
\usepackage{setspace}
\usepackage{fancyhdr}
\usepackage{lastpage}
\usepackage{extramarks}
\usepackage{chngpage}
\usepackage{soul}
\usepackage[usenames,dvipsnames]{color}
\usepackage{graphicx,float,wrapfig}
\usepackage{ifthen}
\usepackage{listings}
\usepackage{courier}
\usepackage{booktabs}
\usepackage{enumerate}
\usepackage{textcomp}
\usepackage{hyperref}

\definecolor{MyDarkGreen}{rgb}{0.0,0.4,0.0}

% For faster processing, load Matlab syntax for listings
\lstloadlanguages{Matlab}%
\lstset{language=Matlab,
        frame=single,
        basicstyle=\small\ttfamily,
        keywordstyle=[1]\color{Blue}\bf,
        keywordstyle=[2]\color{Purple},
        keywordstyle=[3]\color{Blue}\underbar,
        identifierstyle=,
        commentstyle=\usefont{T1}{pcr}{m}{sl}\color{MyDarkGreen}\small,
        stringstyle=\color{Purple},
        showstringspaces=false,
        tabsize=5,
        upquote=true,
        % Put standard MATLAB functions not included in the default
        % language here
        morekeywords={xlim,ylim,var,alpha,factorial,poissrnd,normpdf,normcdf,
        getgenbank,basecount,dimercount,seqshoworfs,cpgisland,fieldnames,getfield,
        codoncount,nt2int},
        % Put MATLAB function parameters here
        morekeywords=[2]{on, off, interp},
        % Put user defined functions here
        morekeywords=[3]{FindESS},
        morecomment=[l][\color{Blue}]{...},
        numbers=left,
        firstnumber=1,
        numberstyle=\tiny\color{Blue},
        stepnumber=5
        }

% In case you need to adjust margins:
\topmargin=-0.45in      %
\evensidemargin=0in     %
\oddsidemargin=0in      %
\textwidth=6.5in        %
\textheight=9.0in       %
\headsep=0.25in         %

% Homework Specific Information
\newcommand{\hmwkTitle}{Exercise 1}
\newcommand{\hmwkSubTitle}{T-61.5120}
\newcommand{\hmwkDueDate}{September 29, 2012}
\newcommand{\hmwkClass}{Computational Genomics}
\newcommand{\hmwkClassTime}{}
\newcommand{\hmwkClassInstructor}{Juho Rousu}
\newcommand{\hmwkAuthorName}{Sakari Cajanus}

% Setup the header and footer
\pagestyle{fancy}                                                       %
\lhead{\hmwkAuthorName}                                                 %
\chead{\hmwkClass\ (\hmwkClassInstructor\ \hmwkClassTime): \hmwkTitle}  %
\rhead{\firstxmark}                                                     %
\lfoot{\lastxmark}                                                      %
\cfoot{}                                                                %
\rfoot{Page\ \thepage\ of\ \protect\pageref{LastPage}}                  %
\renewcommand\headrulewidth{0.4pt}                                      %
\renewcommand\footrulewidth{0.4pt}                                      %

%%%%%%%%%%%%%%%%%%%%%%%%%%%%%%%%%%%%%%%%%%%%%%%%%%%%%%%%%%%%%
% Make title
\title{\vspace{2in}\textmd{\textbf{\hmwkClass:\ \hmwkTitle\ifthenelse{\equal{\hmwkSubTitle}{}}{}{\\\large\hmwkSubTitle}}}\\\normalsize\vspace{0.1in}\small{Due\ on\ \hmwkDueDate}\\\vspace{0.1in}\large{\textit{\hmwkClassInstructor\ \hmwkClassTime}}\vspace{3in}}
\date{}
\author{\textbf{\hmwkAuthorName}}
%%%%%%%%%%%%%%%%%%%%%%%%%%%%%%%%%%%%%%%%%%%%%%%%%%%%%%%%%%%%%

% This is used to trace down (pin point) problems
% in latexing a document:
%\tracingall

%%%%%%%%%%%%%%%%%%%%%%%%%%%%%%%%%%%%%%%%%%%%%%%%%%%%%%%%%%%%%
% Some tools

% Use these to play with the headers and footers
\newcommand{\enterProblemHeader}[1]{\nobreak\extramarks{#1}{#1 continued on next page\ldots}\nobreak%
                                    \nobreak\extramarks{#1 (continued)}{#1 continued on next page\ldots}\nobreak}%
\newcommand{\exitProblemHeader}[1]{\nobreak\extramarks{#1 (continued)}{#1 continued on next page\ldots}\nobreak%
                                   \nobreak\extramarks{#1}{}\nobreak}%

\setcounter{secnumdepth}{0}
\newcommand{\homeworkProblem}[3]
 {\enterProblemHeader{#1}
   \ifthenelse{\equal{#2}{}}%
        {\subsection{#1}}%
        {\subsection{#1 (#2)}}%
               #3
   \exitProblemHeader{#1}}

%%% I think \captionwidth (commented out below) can go away
%%%
%% Edits the caption width
%\newcommand{\captionwidth}[1]{%
%  \dimen0=\columnwidth   \advance\dimen0 by-#1\relax
%  \divide\dimen0 by2
%  \advance\leftskip by\dimen0
%  \advance\rightskip by\dimen0
%}

% Includes a figure
% The first parameter is the label, which is also the name of the figure
%   with or without the extension (e.g., .eps, .fig, .png, .gif, etc.)
%   IF NO EXTENSION IS GIVEN, LaTeX will look for the most appropriate one.
%   This means that if a DVI (or PS) is being produced, it will look for
%   an eps. If a PDF is being produced, it will look for nearly anything
%   else (gif, jpg, png, et cetera). Because of this, when I generate figures
%   I typically generate an eps and a png to allow me the most flexibility
%   when rendering my document.
% The second parameter is the width of the figure normalized to column width
%   (e.g. 0.5 for half a column, 0.75 for 75% of the column)
% The third parameter is the caption.
\newcommand{\scalefig}[3]{
  \begin{figure}[ht!]
    % Requires \usepackage{graphicx}
    \centering
    \includegraphics[width=#2\columnwidth]{#1}
    %%% I think \captionwidth (see above) can go away as long as
    %%% \centering is above
    %\captionwidth{#2\columnwidth}%
    \caption{#3}
    \label{#1}
  \end{figure}}

% Includes a MATLAB script.
% The first parameter is the label, which also is the name of the script
%   without the .m.
% The second parameter is the optional caption.
\newcommand{\matlabscript}[2]
  {\begin{itemize}\item[]\lstinputlisting[caption=#2,label=#1]{#1.m}\end{itemize}}

%%%%%%%%%%%%%%%%%%%%%%%%%%%%%%%%%%%%%%%%%%%%%%%%%%%%%%%%%%%%%

%%%%%%%%%%%%%%%%%%%%%%%%%%%%%%%%%%%%%%%%%%%%%%%%%%%%%%%%%%%%%
%%%%%%%%%% The main document content
%%%%%%%%%%%%%%%%%%%%%%%%%%%%%%%%%%%%%%%%%%%%%%%%%%%%%%%%%%%%%

\begin{document}
\begin{spacing}{1.1}
\maketitle
% Uncomment the \tableofcontents line to get a Contents on front page
%\tableofcontents
\newpage

\homeworkProblem{Problem 1}{}%
{ % Start homework solution
  \begin{enumerate}[(a)]
    \item Dynamic programming matrix:
      \[
      \begin{array}{l|ccccccccc}
        &  & A & G & T & C & T & A & C & G \\
        \hline
        &  0 & -1 & -2 & -3 & -4 & -5 & -6 & -7 & -8\\ 
       G & -1 & -1 & 0 & -1 & -2 & -3 & -4 & -5 & -6 \\
       G & -2 & -2 & 0 & -1 & -2 & -3 & -4 & -5 & -6 \\
       T & -3 & -3 & -1 & +1 & 0 & -1 & -2 & -3 & -4 \\
       G & -4 & -4 & -2 & 0 & 0 & -1 & -2 & -3 & -4 \\
       C & -5 & -5 & -3 & -1 & +1 & 0 & -1 & -1 & -2 \\
       T & -6 & -6 & -4 & -2 & 0 & +2 & +1 & 0 & -1 \\
       C & -7 & -7 & -5 & -3 & -1 & +1 & +1 & +2 & +1 \\
       G & -8 & -8 & -6 & -4 & -2 & 0 & 0 & +1 & +3 \\
      \end{array}
    \]

  \item Trace-back path:
      \[
      \begin{array}{l|ccccccccc}
        &  & A & G & T & C & T & A & C & G \\
        \hline
        &   &  &  &  &  &  &  &  & \\ 
       G &  & \searrow &  &  &  &  &  &  &  \\
       G &  &  & \searrow  &  &  &  &  &  &  \\
       T &  &  &  & \searrow &  &  &  &  &  \\
       G &  &  &  & \downarrow &  &  &  &  &  \\
       C &  &  &  &  & \searrow  &  &  &  &  \\
       T &  &  &  &  &  & \searrow & \rightarrow &  &  \\
       C &  &  &  &  &  &  &  & \searrow &  \\
       G &  &  &  &  &  &  &  &  & \searrow \\
      \end{array}
    \]
  \item Alignment between sequences:\\
    {\tt AGT-CTACG\\
    ||| || ||\\
  GGTGCT-CG\\}
  \end{enumerate}
}
\homeworkProblem{Problem 2}{}%
 {
   \begin{enumerate}[(a)]
     \item{\tt
         TCGT:\\
score = 4.1597\\
align =\\
TCG\\
|||\\
TCG\\
\\
ATGT:\\
score = 1.6639\\
align =\\
ATG\\
| |\\
ACG\\
\\
TCGA:\\
score = 4.1597\\
align =\\
TCG\\
|||\\
TCG}\\
 \item The default substitution matrix \textsc{NUC44} has substitution scores of $+1$ for Cs or Ts to Ys. Making the substitution to the reference we get the following:\\
   {\tt
     TCGT:\\
score = 3.3278\\
align =\\
TCGT\\
|:|:\\
TYGY\\
\\
ATGT:\\
score = 3.3278\\
align =\\
ATGT\\
|:|:\\
AYGY\\
\\
TCGA:\\
score = 3.3278\\
align =\\
TCGA\\
::||\\
YYGA\\
   }
 \item
   {\tt
     TCGT:\\
score = 5.5463\\
align =\\
TTGT\\
||||\\
TTGT\\
\\
ATGT:\\
score = 5.5463\\
align =\\
ATGT\\
||||\\
ATGT\\
\\
TCGA:\\
score = 5.5463\\
align =\\
TTGA\\
||||\\
TTGA\\
   }
   \end{enumerate}
   The first option finds three nucleotide matches for second and last short read, and gives a high score for those. For the second read, only a match of two nucleotides can be found.

   The second options with the wildcard finds matches with the same score for the sequences. Good matches are found, provided the right substitutions have happened in the short reads.

   The third option finds matches with the highest score, as all nucleotides match. Substituting both Ts and Cs with the same letter makes finding the match easier.
 }
 \homeworkProblem{Problem 3}{}%
 {
   Optimal alignment paths:
      \[
      \begin{array}{l|cccccc}
        &  & A & A & C & G & T \\
        \hline
        \\
        A & & \searrow \\
        A & & &  \searrow \\
        C & & & \downarrow & \searrow \\
        G & & & \downarrow & \downarrow & \searrow \\
        C & & & & \searrow\downarrow & \downarrow \\
        G & & & & & \searrow\downarrow \\
        T & & & & & & \searrow 
      \end{array}
    \]
    and alignments:\\
    {\tt
      AA--CGT\\
      || \ |||\\
      AACGCGT\\
\\
      AAC--GT\\
      ||| \ ||\\
      AACGCGT\\
\\
      AACG--T\\
      |||| \ |\\
      AACGCGT\\
 }
 }
 \homeworkProblem{Problem 4}{}%
 {
   \begin{enumerate}[(a)]
     \item The significance level is lower than $0.01 \%$ (i.e. corresponding score (29 or more) would arise in only $0.01\%$ of tests due to chance).
     \item Using only the values given in the table, we can calculate
       \begin{align}
         1-.9999^{1000}=0.09516\dots \approx 9.51\%
       \end{align}
   \end{enumerate}

 }
 \homeworkProblem{Problem 5}{}%
  {
    \begin{enumerate}[(a)]
      \item From \url{http://www.ncbi.nlm.nih.gov/BLAST/blast_program.shtml}
        \begin{enumerate}[1.]
          \item Protein blast
            compares an amino acid query sequence against a protein sequence database
          \item Nucleotide blast
            compares a nucleotide query sequence against a nucleotide sequence database
          \item blastx
            compares a nucleotide query sequence translated in all reading frames against a protein sequence database
          \item tblastn
            compares a protein query sequence against a nucleotide sequence database dynamically translated in all reading frames
          \item tblastx
            compares the six-frame translations of a nucleotide query sequence against the six-frame translations of a nucleotide sequence database. Please note that tblastx program cannot be used with the nr database on the BLAST Web page. 
        \end{enumerate}
      \item 
        \begin{enumerate}[1.]
          \item VecScreen: Finds segment of nucleic acid sequence that may be of vector origin. It searches the query sequence for segments that match UniVec database. \textsc{BLAST} parameters are preset for optimal detection of vector contamination. Finding sequences of vector origin is important to rule out contaminated sequences before submitting them to databases or doing further analysis.
          \item PubChem BioAssay Protein BLAST: searches the BioAssay database that contains biologically active small molecules and their bioactivity results.
          \item Find conserved domains: Searches Conserved Domain Database with protein or nucleotide query sequences. Conserved domains are recurring units in molecular evolution.
        \end{enumerate}
      \item
        \begin{enumerate}[i.]
          \item E-value: Corrected p-value taking into account query and total database sequence lengths.
          \item Bit score $S'$: Normalized version of alignment score: Expected $2^{S'}$ aligments need to be examined to find a bit score of $S'$ by chance.
          \item Identities: The extent to which two sequences have the same residues at the same positions
          \item Query coverage: Percent of the query lengths that is included in the aligned segments.
          \item Total score: Sum of alignment scores of all segments from the same database sequence that match the query sequence (calculated over all segments).
          \item Max score: Highest alignment score (bit score) between the query sequence and the database sequence segment
        \end{enumerate}
    \end{enumerate}
  }
\appendix
\section{Code}
\matlabscript{ex3}{}
\end{spacing}

%\begin{thebibliography}{9}
%\bibitem{hc}
   %NCBI Map Viewer: Homo Sapiens(human),
   %\emph{Chromosome 22}.\\
   %http://www.ncbi.nlm.nih.gov/projects/mapview/maps.cgi?taxid=9607\&chr=22\\
   %retrieved Sept. 18 2012.

%\end{thebibliography}
\end{document}

%%%%%%%%%%%%%%%%%%%%%%%%%%%%%%%%%%%%%%%%%%%%%%%%%%%%%%%%%%%%%

%----------------------------------------------------------------------%
% The following is copyright and licensing information for
% redistribution of this LaTeX source code; it also includes a liability
% statement. If this source code is not being redistributed to others,
% it may be omitted. It has no effect on the function of the above code.
%----------------------------------------------------------------------%
% Copyright (c) 2007, 2008, 2009, 2010, 2011 by Theodore P. Pavlic
%
% Unless otherwise expressly stated, this work is licensed under the
% Creative Commons Attribution-Noncommercial 3.0 United States License. To
% view a copy of this license, visit
% http://creativecommons.org/licenses/by-nc/3.0/us/ or send a letter to
% Creative Commons, 171 Second Street, Suite 300, San Francisco,
% California, 94105, USA.
%
% THE SOFTWARE IS PROVIDED "AS IS", WITHOUT WARRANTY OF ANY KIND, EXPRESS
% OR IMPLIED, INCLUDING BUT NOT LIMITED TO THE WARRANTIES OF
% MERCHANTABILITY, FITNESS FOR A PARTICULAR PURPOSE AND NONINFRINGEMENT.
% IN NO EVENT SHALL THE AUTHORS OR COPYRIGHT HOLDERS BE LIABLE FOR ANY
% CLAIM, DAMAGES OR OTHER LIABILITY, WHETHER IN AN ACTION OF CONTRACT,
% TORT OR OTHERWISE, ARISING FROM, OUT OF OR IN CONNECTION WITH THE
% SOFTWARE OR THE USE OR OTHER DEALINGS IN THE SOFTWARE.
%----------------------------------------------------------------------%
