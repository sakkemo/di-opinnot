\documentclass{article}
% Change "article" to "report" to get rid of page number on title page
\usepackage{fontenc}
\usepackage{amsmath,amsfonts,amsthm,amssymb}
\usepackage{setspace}
\usepackage{fancyhdr}
\usepackage{lastpage}
\usepackage{extramarks}
\usepackage{chngpage}
\usepackage{soul}
\usepackage[usenames,dvipsnames]{color}
\usepackage{graphicx,float,wrapfig}
\usepackage{ifthen}
\usepackage{listings}
\usepackage{courier}
\usepackage{booktabs}
\usepackage{enumerate}
\usepackage{textcomp}

\definecolor{MyDarkGreen}{rgb}{0.0,0.4,0.0}

% For faster processing, load Matlab syntax for listings
\lstloadlanguages{Matlab}%
\lstset{language=Matlab,
        frame=single,
        basicstyle=\small\ttfamily,
        keywordstyle=[1]\color{Blue}\bf,
        keywordstyle=[2]\color{Purple},
        keywordstyle=[3]\color{Blue}\underbar,
        identifierstyle=,
        commentstyle=\usefont{T1}{pcr}{m}{sl}\color{MyDarkGreen}\small,
        stringstyle=\color{Purple},
        showstringspaces=false,
        tabsize=5,
        upquote=true,
        % Put standard MATLAB functions not included in the default
        % language here
        morekeywords={xlim,ylim,var,alpha,factorial,poissrnd,normpdf,normcdf,
        getgenbank,basecount,dimercount,seqshoworfs,cpgisland,fieldnames,getfield,
        codoncount,nt2int},
        % Put MATLAB function parameters here
        morekeywords=[2]{on, off, interp},
        % Put user defined functions here
        morekeywords=[3]{FindESS},
        morecomment=[l][\color{Blue}]{...},
        numbers=left,
        firstnumber=1,
        numberstyle=\tiny\color{Blue},
        stepnumber=5
        }

% In case you need to adjust margins:
\topmargin=-0.45in      %
\evensidemargin=0in     %
\oddsidemargin=0in      %
\textwidth=6.5in        %
\textheight=9.0in       %
\headsep=0.25in         %

% Homework Specific Information
\newcommand{\hmwkTitle}{Exercise 1}
\newcommand{\hmwkSubTitle}{T-61.5120}
\newcommand{\hmwkDueDate}{September 16, 2012}
\newcommand{\hmwkClass}{Computational Genomics}
\newcommand{\hmwkClassTime}{}
\newcommand{\hmwkClassInstructor}{Juho Rousu}
\newcommand{\hmwkAuthorName}{Sakari Cajanus}

% Setup the header and footer
\pagestyle{fancy}                                                       %
\lhead{\hmwkAuthorName}                                                 %
\chead{\hmwkClass\ (\hmwkClassInstructor\ \hmwkClassTime): \hmwkTitle}  %
\rhead{\firstxmark}                                                     %
\lfoot{\lastxmark}                                                      %
\cfoot{}                                                                %
\rfoot{Page\ \thepage\ of\ \protect\pageref{LastPage}}                  %
\renewcommand\headrulewidth{0.4pt}                                      %
\renewcommand\footrulewidth{0.4pt}                                      %

%%%%%%%%%%%%%%%%%%%%%%%%%%%%%%%%%%%%%%%%%%%%%%%%%%%%%%%%%%%%%
% Make title
\title{\vspace{2in}\textmd{\textbf{\hmwkClass:\ \hmwkTitle\ifthenelse{\equal{\hmwkSubTitle}{}}{}{\\\large\hmwkSubTitle}}}\\\normalsize\vspace{0.1in}\small{Due\ on\ \hmwkDueDate}\\\vspace{0.1in}\large{\textit{\hmwkClassInstructor\ \hmwkClassTime}}\vspace{3in}}
\date{}
\author{\textbf{\hmwkAuthorName}}
%%%%%%%%%%%%%%%%%%%%%%%%%%%%%%%%%%%%%%%%%%%%%%%%%%%%%%%%%%%%%

% This is used to trace down (pin point) problems
% in latexing a document:
%\tracingall

%%%%%%%%%%%%%%%%%%%%%%%%%%%%%%%%%%%%%%%%%%%%%%%%%%%%%%%%%%%%%
% Some tools

% Use these to play with the headers and footers
\newcommand{\enterProblemHeader}[1]{\nobreak\extramarks{#1}{#1 continued on next page\ldots}\nobreak%
                                    \nobreak\extramarks{#1 (continued)}{#1 continued on next page\ldots}\nobreak}%
\newcommand{\exitProblemHeader}[1]{\nobreak\extramarks{#1 (continued)}{#1 continued on next page\ldots}\nobreak%
                                   \nobreak\extramarks{#1}{}\nobreak}%

\setcounter{secnumdepth}{0}
\newcommand{\homeworkProblem}[3]
 {\enterProblemHeader{#1}
   \ifthenelse{\equal{#2}{}}%
        {\subsection{#1}}%
        {\subsection{#1 (#2)}}%
               #3
   \exitProblemHeader{#1}}

%%% I think \captionwidth (commented out below) can go away
%%%
%% Edits the caption width
%\newcommand{\captionwidth}[1]{%
%  \dimen0=\columnwidth   \advance\dimen0 by-#1\relax
%  \divide\dimen0 by2
%  \advance\leftskip by\dimen0
%  \advance\rightskip by\dimen0
%}

% Includes a figure
% The first parameter is the label, which is also the name of the figure
%   with or without the extension (e.g., .eps, .fig, .png, .gif, etc.)
%   IF NO EXTENSION IS GIVEN, LaTeX will look for the most appropriate one.
%   This means that if a DVI (or PS) is being produced, it will look for
%   an eps. If a PDF is being produced, it will look for nearly anything
%   else (gif, jpg, png, et cetera). Because of this, when I generate figures
%   I typically generate an eps and a png to allow me the most flexibility
%   when rendering my document.
% The second parameter is the width of the figure normalized to column width
%   (e.g. 0.5 for half a column, 0.75 for 75% of the column)
% The third parameter is the caption.
\newcommand{\scalefig}[3]{
  \begin{figure}[ht!]
    % Requires \usepackage{graphicx}
    \centering
    \includegraphics[width=#2\columnwidth]{#1}
    %%% I think \captionwidth (see above) can go away as long as
    %%% \centering is above
    %\captionwidth{#2\columnwidth}%
    \caption{#3}
    \label{#1}
  \end{figure}}

% Includes a MATLAB script.
% The first parameter is the label, which also is the name of the script
%   without the .m.
% The second parameter is the optional caption.
\newcommand{\matlabscript}[2]
  {\begin{itemize}\item[]\lstinputlisting[caption=#2,label=#1]{#1.m}\end{itemize}}

%%%%%%%%%%%%%%%%%%%%%%%%%%%%%%%%%%%%%%%%%%%%%%%%%%%%%%%%%%%%%

%%%%%%%%%%%%%%%%%%%%%%%%%%%%%%%%%%%%%%%%%%%%%%%%%%%%%%%%%%%%%
%%%%%%%%%% The main document content
%%%%%%%%%%%%%%%%%%%%%%%%%%%%%%%%%%%%%%%%%%%%%%%%%%%%%%%%%%%%%

\begin{document}
\begin{spacing}{1.1}
\maketitle
% Uncomment the \tableofcontents line to get a Contents on front page
%\tableofcontents
\newpage

\homeworkProblem{Problem 1}{}%
 { % Start homework solution

 \begin{enumerate}[(a)]
\item The
\begin{align}
\textrm{GC-content}=
\cfrac{G+C}{A+T+G+C}\times\ 100 \%,
\end{align}
was 49.1 \%.
\item Figure~\ref{fig:awesome_image}
    \begin{figure}[h]
\centering
\includegraphics[width=10cm]{1b}
\caption{GC-skew}
\label{fig:awesome_image}
\end{figure}
\item Base frequencies
%
\begin{table}[h!]
\caption{Base frequencies\vspace{2pt}}
\centering
\begin{tabular}{llr}
\toprule
base (upper strand) & base (lower strand) & frequecny \\
\hline
A & T & 14 \\
C & G & 16 \\
G & C & 12 \\
T & A & 15 \\
\bottomrule
\end{tabular}
\end{table}
\item Dinucleotide frequencies
%
\begin{table}[h!]
\caption{Dinucleotide frequencies\vspace{2pt}}
\centering
\begin{tabular}{lr}
\toprule
Dinucleotide (upper strand) & frequency \\
\hline
    AA & 3\\
    AC & 4\\
    AG & 5\\
    AT & 3\\
    CA & 7\\
    CC & 0\\
    CG & 1\\
    CT & 4\\
    GA & 5\\
    GC & 2\\
    GG & 6\\
    GT & 3\\
    TA & 0\\
    TC & 6\\
    TG & 3\\
    TT & 4\\
\bottomrule
\end{tabular}
\end{table}
\item Trinucleotide frequencies
\begin{table}[h!]
\caption{Trinucleotide frequencies\vspace{2pt}}
\centering
\begin{tabular}{lr}
\toprule
Trinucleotide (upper strand) & frequency \\
\hline
 AAA    & 1\\
 AAC    & 2\\
 ACA    & 3\\
 ACT    & 1\\
 AGA    & 1\\
 AGG    & 2\\
 AGT    & 2\\
 ATC    & 2\\
 ATT    & 1\\
 CAA    & 2\\
 CAC    & 1\\
 CAG    & 1\\
 CAT    & 3\\
 CGC    & 1\\
 CTC    & 1\\
 CTG    & 2\\
 CTT    & 1\\
 GAC    & 1\\
 GAG    & 4\\
 GCA    & 1\\
 GCT    & 1\\
 GGA    & 3\\
 GGG    & 2\\
 GGT    & 1\\
 GTC    & 1\\
 GTT    & 1\\
 TCA    & 3\\
 TCG    & 1\\
 TCT    & 2\\
 TGA    & 1\\
 TGC    & 1\\
 TGG    & 1\\
 TTC    & 2\\
 TTG    & 1\\
 TTT    & 1\\
\bottomrule
\end{tabular}
\end{table}
\end{enumerate}
 } % End homework solution
\clearpage
\homeworkProblem{Problem 2}{Codon Adaptation Index}{
    log-odd CAIs were:
\begin{enumerate}[(a)]
    \item $-0.4678$
    \item $-0.2713$
\end{enumerate}
The latter is more likely from a highly expressed gene.
}
\newpage
\homeworkProblem{Problem 3}{1st Order Markov Model}{
Given the frequencies and transition probabilities, the log-probabilities under the first order Markov chain model were
\begin{enumerate}[(a)]
    \item $-43.9689$
    \item $-43.0756$
\end{enumerate}
and given only the base frequencies
\begin{enumerate}[(a)]
    \item $-43.2978$
    \item $-42.7827$
\end{enumerate}
The sequence (b) was more likely to originate from the first order Markov chain model.
}
\newpage
\homeworkProblem{Problem 4}{Over- and under-represented k-mers}{
\begin{table}[ht]
\caption{K-mers\vspace{2pt}}
\centering
\begin{tabular}{lrrrl}
\toprule
k-mer & frequency & probability & odds ratio & over-presented? \\
\hline
 C      &  0.3   &   0.2450     &  1.2245    &  X\\
 AAA    &  0.01  &   0.0140     &  0.7144    &   \\
 CGCG   &  0.001 &   0.0045     &  0.2235    &    \\
 TACG   &  0.005 &   0.0039     &  1.2871    &  X  \\
 GGG    &  0.01  &   0.0203     &  0.4915    &   \\
 TATATA &  0.001 &   0.0002     &  5.1038    &   X \\
    
    
    
    
    
      
\bottomrule
\end{tabular}
\end{table}
}
\homeworkProblem{Problem 5}{Over- and under-represented k-mers}{
    \begin{enumerate}[(a)]
        \item 3 similarities:
            \begin{itemize}
                \item Free access/data publicly available
                \item All allow searching/alignment with BLAST
                \item All have APIs for software/batch retrieval of data
            \end{itemize}
        \item 3 differences:
            \begin{itemize}
                \item Different data sources
                \item Focusing on different things: genome data/all molecular data/proteins, respectively
                \item Different financial backers (at least partly)
            \end{itemize}
        \item Searching for GFP:
            \begin{itemize}
                \item Genbank gives genetic sequences; EMBL sequences as well as molecules of various sizes, gene expression, pathways...; uniprot has protein structure and amino acid sequences
                \item Most results: EMBL, least results: uniprot. EMBL has various information sources and does not focus on only genetic sequences or proteins.
                \item GFP has been studied extensively, and made synthetically
                \item Some results are the same or have the same access numbers etc.
            \end{itemize}
        \item Green fluorescent protein is originally from jellyfish. It has been used as noninvasive fluorescent marker extensively.
    \end{enumerate}
}
\appendix
\section{Code}
\matlabscript{ex1}{}
\end{spacing}

%\begin{thebibliography}{9}
%\bibitem{hc}
   %NCBI Map Viewer: Homo Sapiens(human),
   %\emph{Chromosome 22}.\\
   %http://www.ncbi.nlm.nih.gov/projects/mapview/maps.cgi?taxid=9607\&chr=22\\
   %retrieved Sept. 18 2012.

%\end{thebibliography}
\end{document}

%%%%%%%%%%%%%%%%%%%%%%%%%%%%%%%%%%%%%%%%%%%%%%%%%%%%%%%%%%%%%

%----------------------------------------------------------------------%
% The following is copyright and licensing information for
% redistribution of this LaTeX source code; it also includes a liability
% statement. If this source code is not being redistributed to others,
% it may be omitted. It has no effect on the function of the above code.
%----------------------------------------------------------------------%
% Copyright (c) 2007, 2008, 2009, 2010, 2011 by Theodore P. Pavlic
%
% Unless otherwise expressly stated, this work is licensed under the
% Creative Commons Attribution-Noncommercial 3.0 United States License. To
% view a copy of this license, visit
% http://creativecommons.org/licenses/by-nc/3.0/us/ or send a letter to
% Creative Commons, 171 Second Street, Suite 300, San Francisco,
% California, 94105, USA.
%
% THE SOFTWARE IS PROVIDED "AS IS", WITHOUT WARRANTY OF ANY KIND, EXPRESS
% OR IMPLIED, INCLUDING BUT NOT LIMITED TO THE WARRANTIES OF
% MERCHANTABILITY, FITNESS FOR A PARTICULAR PURPOSE AND NONINFRINGEMENT.
% IN NO EVENT SHALL THE AUTHORS OR COPYRIGHT HOLDERS BE LIABLE FOR ANY
% CLAIM, DAMAGES OR OTHER LIABILITY, WHETHER IN AN ACTION OF CONTRACT,
% TORT OR OTHERWISE, ARISING FROM, OUT OF OR IN CONNECTION WITH THE
% SOFTWARE OR THE USE OR OTHER DEALINGS IN THE SOFTWARE.
%----------------------------------------------------------------------%
