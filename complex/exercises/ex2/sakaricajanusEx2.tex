\documentclass[a4paper]{article}
\usepackage[english]{babel}
\usepackage{ucs}
\usepackage[utf8x]{inputenc}
\usepackage[T1]{fontenc}
\usepackage{times}
\usepackage[pdftex]{graphicx}
\usepackage{color}
\usepackage[pdftex,colorlinks=true,citecolor=black,
            pagecolor=black,linkcolor=black,menucolor=black,
            urlcolor=black]{hyperref}
\usepackage{eufrak}
\usepackage{amsmath}
\usepackage{amsbsy}
\usepackage{eucal}
\usepackage{subfigure}
\usepackage{longtable}
\usepackage{url}
\usepackage{enumerate}
\urlstyle{same}

\usepackage{natbib}

\pdfinfo{
          /Title      (BECS-114.4150 Complex Networks)
          /Author     ()
          /Keywords   ()
}


\title{ BECS-114.4150 Complex Networks - Problem set 2}
\author{Sakari Cajanus\\ 82036R \\
       {\it sakari.cajanus@aalto.fi}}


\begin{document}
%--- KANSILEHTI -------------------------------------------------------------
\maketitle


\newpage
%------------------------------------------------------------------------------

\section*{Problem 1}
\begin{enumerate}[i)]
    \item Download datasets (done)
    \item Visualizations\\
        For the dolphins, it seems the clustering coefficient would be low for a majority of the nodes.\\
        Conversely, for the Zachary karate club network the clustering coefficient is high for majority.

        
\end{enumerate}
\clearpage
\section*{Problem 2}
I simulated a Markov chain of length 1000, with initial distribution of
$\pi = (\begin{smallmatrix} 0.5 & 0.5\end{smallmatrix})$ and transition matrix  $P = \bigl(\begin{smallmatrix} 0.9 &
0.1 \\ 0.2 & 0.8\end{smallmatrix}\bigr)$.
In this case, the stationary distribution (where the chain would converge after run-
ning for infinite time) would be $\pi = (\begin{smallmatrix} 2/3 & 1/3 \end{smallmatrix})$. For the thousand steps, the ratio of
the time spent in the states was $\pi = (\begin{smallmatrix}  0.644 & 0.356 \end{smallmatrix})$, so the chain is quite close to the
stationary distribution after it has run for 1000 steps. The plot of the path
shows how the states switch or stay the same: more time is spent in state 1.
\begin{figure}[ht]
    \centering
     %\includegraphics[width=0.85\textwidth]{disc.pdf}
\end{figure}
\clearpage
\section*{Problem 3}
I simulated a Markov chain of length 20 time units, with initial distribution of
$\pi = (\begin{smallmatrix} 1 & 0\end{smallmatrix})$ and transition rate matrix 
$Q = \bigl(\begin{smallmatrix} -0.5 & 0.5 \\ 1 & -1\end{smallmatrix}\bigr)$.

Here, again, the Markov chain favours state 1 and spends more time there (which
is apparent from the transition rate matrix as well: The rate of transition
from state 1 to state 2 is
smaller).
\begin{figure}[ht]
    \centering
     %\includegraphics[width=0.85\textwidth]{cont.pdf}
\end{figure}

\end{document}
%%% Local Variables:
%%% mode: latex
%%% TeX-master: t
%%% End:
