\documentclass{beamer}
\usepackage[english]{babel}
\usepackage{booktabs}
\mode<presentation>
\usepackage[utf8]{inputenc}
\usepackage{graphicx}
\usepackage{amsmath}
%\usetheme{Warsaw}
%\usetheme{Marburg}
\usetheme{Goettingen}
%\usetheme{Hannover}
%\usetheme{Berkeley}
%\usetheme{Malmoe}
\usecolortheme{crane}
%\usecolortheme{wolverine}
\setbeamercolor{sidebar canvas left}{fg=orange}
\setbeamercolor{sidebar}{parent=palette crane}
%\usetheme{Copenhagen}

\author{Sakari Cajanus}
\title[Computational Systems Biology]{Epidemic modelling}
%\subtitle{\small Aaron Arvey, Phaedra Agius, William Stafford Noble et al.}
\institute{Aalto University}
\date{January 21, 2012}
\AtBeginSubsection[]
{
  \begin{frame}<beamer>
    \frametitle{Layout}
    \tableofcontents[currentsection,currentsubsection]
  \end{frame}
}
\setbeamercovered{transparent}

\begin{document}

\begin{frame}
\titlepage
\end{frame}

\section{Contents}
\begin{frame}{Contents}
\begin{enumerate}
\item Motivation: Real-life example (H5N1 in 2005)
\item Definition of epidemic modelling and a little bit of history
\item Types of models used in epidemic modelling
\item Few simple models explained in detail
\end{enumerate}
\end{frame}

\section{Motivation}
\begin{frame}{Notable examples}
Avian flu (2005):
\begin{itemize}
\item Two research papers: Nature (August 3) and Science (August 5)
\item Simulations were used to see what would happen if avian flu were to start passing efficiently between humans
\item Both teams showed that antiviral treatments were critical in containing the pandemic
\end{itemize}
\end{frame}

\begin{frame}{H5N1 in 2005}
\begin{itemize}
\item In 2005, the avian flu was endemic in Southeast Asian avian population
\item As well as incapable of human-to-human transmission
\item The H5N1 represents a serious pandemic threat, should it acquire human-to-human transmission capability through mutation or reassortment
\item The model in \emph{Nature} was used to simulate the effectiveness of containment and mass use of prophylactic antiviral drugs using different distribution strategies, assuming different values for \emph{reproduction number} of the virus
\end{itemize}
\end{frame}

\begin{frame}{Reproduction Number $R_0$}
Reproduction number $R_0$ represents the average number of secondary cases generated by a primary case in the entire suspectible population.

\begin{itemize}
\item $R_0 < 1$: The chains of transmission will inevitably die out
\item $R_0 > 1$: The disease can spread in the population
\end{itemize}

Hence, the goal of control policies is to reduce $R_0$ below 1.
\end{frame}

\begin{frame}{Simulated Public Health Policies}
The goal of the study was identifying public health policies that might stop the pandemic from spreading.
\begin{itemize}
\item New vaccines might not be effective, and would need to target large number of people
\item \textbf{Antiviral drugs} can be used to contain the pandemic by reducing transmission rate
\item As having large stockpiles of drugs is impossible, containing the pandemic using antivirals if only possible at the source of the nascent pandemic
\end{itemize}
The study focused on the means of using targeted antivirals and social distancing measures for containing the pandemic.
\end{frame}

\begin{frame}{Model}
The model used in the study was used to simulate spread of the pandemic in Thailand and neighboring countries, as that was where suitable information was available.
\begin{itemize}
\item 85 million people
\item The model was spatially explicit, and took into account school and workplace sizes as well as day-to-day movement and travel
\item In absence of H5-based pandemic strain data, past human influenza were used for parameter estimates of the disease: Infectiousness was also modelled using as a function of time
\end{itemize}
\end{frame}

\begin{frame}{Model}
Transmission and antiviral drug action:
\begin{itemize}
\item The transmission model was stochastic and individual-based, taking into account age, generational structure within households and household size. Households were randomly distributed with a local density based on geographic data.
\item Infection risk was modelled separately from 3 sources: (1) household, (2) place and (3) random encounters
\item Recent statistical estimates were used for the antiviral drug efficiency
\end{itemize}
\end{frame}


\begin{frame}{Conclusions (1)}
\href{http://www.nigms.nih.gov/News/Results/08032005.htm}{http://www.nigms.nih.gov/News/Results/08032005.htm}
\begin{itemize}
\item Blanket prophylaxis of the entire country or region should be able to eliminate pandemic virus with an $R_0$ of 3.6
\item Social targeting of the drug alone would only have $\geq 90 \%$ probability of eliminating the pandemic strain, if $R_0 \leq 1.25$, if initiated after 20 or more cases
\item Geographic targeting, while costly, is also effective: With two-day delay from onset to prophylaxis, a 5-km ring policy is able to contain pandemics with an $R_0$ of 1.5 at the cost of 2 million courses of the drug
\end{itemize}
\end{frame}

\begin{frame}{Conclusions (2)}
Containment and elimination is possible, if combination of antiviral drugs and social distancing measures are used. However, the following conditions must be met:

\begin{enumerate}
\item Rapid identification of original cluster
\item Rapid case detection and delivery of treatment ($<48$ hours)
\item Effective delivery of treatment to high proportion of targeted population ($>$ 90 \%)
\item Sufficient stockpiles of drugs, preferebly 3M or more
\item Population cooperation
\item International cooperation in policy development
\end{enumerate}
\end{frame}


\section{Definition}

\begin{frame}{Epidemic modelling}
\begin{itemize}
\item An epidemic model is a simplified means of describing the transmission of communicable disease through individuals (Wikipedia)
\item Epidemic modelling can be used to study mechanisms by which diseases spread, as well as to predict the course of an outbreak or to evaluate strategies to control the epidemic, such as inoculation (vaccination or purposeful infection) or isolation against the disease
\item Stopping a global pandemic is easier when steps are taken as early as possible
\end{itemize}
\end{frame}

\begin{frame}{History}
\begin{columns}
\begin{column}{.60\textwidth}
\begin{itemize}
\item Earliest account of mathematical model was for practice of inoculation against smallpox (Daniel Bernoulli, 1766)
\item In 1927, a simple deterministic model with three groups was introduced and was used to succesfully predict behaviour of outbreaks similar to observed and recorded epidemics
\end{itemize}
\end{column}%
\hfill%
\begin{column}{.38\textwidth}
\begin{figure}[ht]
  \begin{center}
      \includegraphics[width=0.8\textwidth]{Daniel_Bernoulli_001}
    \caption{Daniel Bernoulli}\label{bernoulli}
 \end{center}
\end{figure}
\end{column}%
\end{columns}
\end{frame}

\section{Types of Models}
\begin{frame}{Types of Epidemic Models}
\begin{enumerate}
\item Deterministic
\begin{figure}[ht]
  \begin{center}
      \includegraphics[width=0.8\textwidth]{SIR.PNG}
    \caption{A very simple compartmental model}\label{bernoulli}
 \end{center}
\end{figure}
\item Stochastic
\end{enumerate}
\end{frame}

\begin{frame}{Deterministic models}
\begin{itemize}
\item Deterministic or compartmental models are suitable for modelling large populations, where fluctuations due to chance are not as significant
\item Individuals are assigned to different subgroups: Suspectible, Infected and Resistant, for example
\item Transition rates are described as derivatives, so the model can be described using differential equations
\item Another approach would be discrete analysis on a lattice
% Piirrä kuva: S->R->I
\end{itemize}
\end{frame}


\begin{frame}{Stochastic models}
\begin{itemize}
\item Stochastic models need to be used when populations are small
\item Models depend on chance variations in parameters such as risk of exposure or infectious period
\item Some stochastic properties are intrinsic in the modelled phenomena: The infection rate, for example, depends on the integer number of disease carrying organisms in the host as well as
\end{itemize}
\end{frame}
%SUMMA SUMMARUM /todo

\begin{frame}{Examples of Models}
\begin{enumerate}
\item SIR model
\item Compartmental model for bird flu
\item A simple stochastic model
\end{enumerate}
\end{frame}

\section{Examples}
\begin{frame}{SIR Model (Deterministic)}
The SIR model from 1927 is probably the simplest usable model. It has a fixed population with three compartments:
\begin{itemize}
\item $S(t)$: number of individuals not yet infected at time $t$
\item $I(t)$: number of individuals infected and capable of spreading the disease
\item $R(t)$: those individuals who have recovered, and are neither suspectible or capable of spreading the disease
\end{itemize}
\end{frame}

\begin{frame}{SIR Model}
The SIR model can be expressed using following set of differential equations:
\begin{align*}
&\frac{dS}{dt} = - \beta S I\\
&\frac{dI}{dt} = \beta S I - \gamma I \\
&\frac{dR}{dt} = \gamma I,
\end{align*}
where $\beta$ is the contact rate and $1/\gamma$ is the average infectious period.
\end{frame}


\begin{frame}{SIR Model Dynamic}

\end{frame}

\begin{frame}{Modelling Bird Flu}
\end{frame}


\end{document}

