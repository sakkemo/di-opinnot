\documentclass{article}
% Change "article" to "report" to get rid of page number on title page
\usepackage{fontenc}
\usepackage{amsmath,amsfonts,amsthm,amssymb}
\usepackage{setspace}
\usepackage{fancyhdr}
\usepackage{lastpage}
\usepackage{extramarks}
\usepackage{chngpage}
\usepackage{soul}
\usepackage[usenames,dvipsnames]{color}
\usepackage{graphicx,float,wrapfig}
\usepackage{ifthen}
\usepackage{listings}
\usepackage{courier}
\usepackage{booktabs}
\usepackage{enumerate}
\usepackage{textcomp}
\usepackage{hyperref}
\usepackage{caption,subcaption}
\usepackage{enumerate}

\definecolor{MyDarkGreen}{rgb}{0.0,0.4,0.0}

% For faster processing, load Matlab syntax for listings
\lstloadlanguages{Matlab}%
\lstset{language=Matlab,
        frame=single,
        basicstyle=\small\ttfamily,
        keywordstyle=[1]\color{Blue},
        %keywordstyle=[1]\color{Blue}\bf,
        keywordstyle=[2]\color{Purple},
        keywordstyle=[3]\color{Blue}\underbar,
        identifierstyle=,
        commentstyle=\usefont{T1}{pcr}{m}{sl}\color{MyDarkGreen}\small,
        stringstyle=\color{Purple},
        showstringspaces=false,
        tabsize=5,
        upquote=true,
        % Put standard MATLAB functions not included in the default
        % language here
        morekeywords={xlim,ylim,var,alpha,factorial,poissrnd,normpdf,normcdf,
        getgenbank,basecount,dimercount,seqshoworfs,cpgisland,fieldnames,getfield,
        codoncount,nt2int,meansqr,double,imread,imshow,ceil,noise,medfilt2,
        fspecial,imwrite},
        % Put MATLAB function parameters here
        morekeywords=[2]{on, off, interp},
        % Put user defined functions here
        morekeywords=[3]{FindESS},
        morecomment=[l][\color{Blue}]{...},
        numbers=left,
        firstnumber=1,
        numberstyle=\tiny\color{Blue},
        stepnumber=5
        }

% In case you need to adjust margins:
\topmargin=-0.45in      %
\evensidemargin=0in     %
\oddsidemargin=0in      %
\textwidth=6.5in        %
\textheight=9.0in       %
\headsep=0.25in         %

% Homework Specific Information
\newcommand{\hmwkTitle}{Assignment 2}
\newcommand{\hmwkSubTitle}{T-61.5100}
\newcommand{\hmwkDueDate}{November 19, 2013}
\newcommand{\hmwkClass}{Digital Image Processing P}
\newcommand{\hmwkClassTime}{}
\newcommand{\hmwkClassInstructor}{}
\newcommand{\hmwkAuthorName}{Sakari Cajanus, 82036R}

% Setup the header and footer
\pagestyle{fancy}                                                       %
\lhead{\hmwkAuthorName}                                                 %
\chead{\hmwkClass\ (\hmwkClassInstructor\ \hmwkClassTime): \hmwkTitle}  %
\rhead{\firstxmark}                                                     %
\lfoot{\lastxmark}                                                      %
\cfoot{}                                                                %
\rfoot{Page\ \thepage\ of\ \protect\pageref{LastPage}}                  %
\renewcommand\headrulewidth{0.4pt}                                      %
\renewcommand\footrulewidth{0.4pt}                                      %

%%%%%%%%%%%%%%%%%%%%%%%%%%%%%%%%%%%%%%%%%%%%%%%%%%%%%%%%%%%%%
% Make title
\title{\vspace{2in}\textmd{\textbf{\hmwkClass:\ \hmwkTitle\ifthenelse{\equal{\hmwkSubTitle}{}}{}{\\\large\hmwkSubTitle}}}\\\normalsize\vspace{0.1in}\small{Due\ on\ \hmwkDueDate}\\\vspace{0.1in}\large{\textit{\hmwkClassInstructor\ \hmwkClassTime}}\vspace{3in}}
\date{}
\author{\textbf{\hmwkAuthorName}}
%%%%%%%%%%%%%%%%%%%%%%%%%%%%%%%%%%%%%%%%%%%%%%%%%%%%%%%%%%%%%

% This is used to trace down (pin point) problems
% in latexing a document:
%\tracingall

%%%%%%%%%%%%%%%%%%%%%%%%%%%%%%%%%%%%%%%%%%%%%%%%%%%%%%%%%%%%%
% Some tools

% Use these to play with the headers and footers
\newcommand{\enterProblemHeader}[1]{\nobreak\extramarks{#1}{#1 continued on next page\ldots}\nobreak%
                                    \nobreak\extramarks{#1 (continued)}{#1 continued on next page\ldots}\nobreak}%
\newcommand{\exitProblemHeader}[1]{\nobreak\extramarks{#1 (continued)}{#1 continued on next page\ldots}\nobreak%
                                   \nobreak\extramarks{#1}{}\nobreak}%

\setcounter{secnumdepth}{0}
\newcommand{\homeworkProblem}[3]
 {\enterProblemHeader{#1}
   \ifthenelse{\equal{#2}{}}%
        {\subsection{#1}}%
        {\subsection{#1 (#2)}}%
               #3
   \exitProblemHeader{#1}}

%%% I think \captionwidth (commented out below) can go away
%%%
%% Edits the caption width
%\newcommand{\captionwidth}[1]{%
%  \dimen0=\columnwidth   \advance\dimen0 by-#1\relax
%  \divide\dimen0 by2
%  \advance\leftskip by\dimen0
%  \advance\rightskip by\dimen0
%}

% Includes a figure
% The first parameter is the label, which is also the name of the figure
%   with or without the extension (e.g., .eps, .fig, .png, .gif, etc.)
%   IF NO EXTENSION IS GIVEN, LaTeX will look for the most appropriate one.
%   This means that if a DVI (or PS) is being produced, it will look for
%   an eps. If a PDF is being produced, it will look for nearly anything
%   else (gif, jpg, png, et cetera). Because of this, when I generate figures
%   I typically generate an eps and a png to allow me the most flexibility
%   when rendering my document.
% The second parameter is the width of the figure normalized to column width
%   (e.g. 0.5 for half a column, 0.75 for 75% of the column)
% The third parameter is the caption.
\newcommand{\scalefig}[3]{
  \begin{figure}[ht!]
    % Requires \usepackage{graphicx}
    \centering
    \includegraphics[width=#2\columnwidth]{#1}
    %%% I think \captionwidth (see above) can go away as long as
    %%% \centering is above
    %\captionwidth{#2\columnwidth}%
    \caption{#3}
    \label{#1}
  \end{figure}}

% Includes a MATLAB script.
% The first parameter is the label, which also is the name of the script
%   without the .m.
% The second parameter is the optional caption.
\newcommand{\matlabscript}[2]
  {\begin{itemize}\item[]\lstinputlisting[caption=#2,label=#1]{#1.m}\end{itemize}}

%%%%%%%%%%%%%%%%%%%%%%%%%%%%%%%%%%%%%%%%%%%%%%%%%%%%%%%%%%%%%

%%%%%%%%%%%%%%%%%%%%%%%%%%%%%%%%%%%%%%%%%%%%%%%%%%%%%%%%%%%%%
%%%%%%%%%% The main document content
%%%%%%%%%%%%%%%%%%%%%%%%%%%%%%%%%%%%%%%%%%%%%%%%%%%%%%%%%%%%%

\begin{document}
\begin{spacing}{1.1}
\maketitle
% Uncomment the \tableofcontents line to get a Contents on front page
%\tableofcontents
\newpage

\homeworkProblem{}{}%
{ % Start homework solution
    In collaboration with Betti Valmari.

    We were supposed to filter sinusoidal noise pattern from the image using
    two different methods: First, a spatial model (finding a model for the
    inference pattern) and then by removing frequency domain spikes from FFT of
    the image. The original and noisy images are shown in figure~\ref{fig:gull}
    and~\ref{fig:tiger}, respectively.
\begin{figure}
        \centering
        \begin{subfigure}[b]{0.5\textwidth}
                \includegraphics[width=\textwidth]{T-61_5100_city_orig_2.png}
                \caption{city\_orig.png}
                \label{fig:gull}
        \end{subfigure}%
        ~ %add desired spacing between images, e. g. ~, \quad, \qquad etc.
          %(or a blank line to force the subfigure onto a new line)
        \begin{subfigure}[b]{0.5\textwidth}
                \includegraphics[width=\textwidth]{T-61_5100_city_lines}
                \caption{city\_lines.png}
                \label{fig:tiger}
        \end{subfigure}
        \caption{Original and noisy images.}\label{fig:animals}
\end{figure}

To start, we were asked to calculate the mean square error before and after the
operations.
The mean square error is calculated by taking the difference (signifying
\emph{error}) in the intensity of each pixel, squaring it and then calculating
the mean over all the pixels in the quadrant. In general, the formula is
\begin{align}
    \operatorname{MSE}=\frac{1}{n}\sum_{i=1}^n(\hat{Y_i} - Y_i)^2,
\end{align}
where $\hat{Y}$ are the values measured and $Y$ are the true values. We will
also show the filtered images as well as the error images (filtered image
substracted from the original) to visually compare the results.

The MSE values were calculated using intensities normalized between 0 and 1. MSEs
can be calculated in Matlab either by calculating the differences, squares and
means by basic operations or using the function {\tt meansqr()}.

\begin{description}
    \item[Spatial model]
We did the spatial filtering using a model with four free parameters $A_1$,
$A_2$, $\delta_1$ and $\delta_2$. The frequency (number of maximums in the
image) of the sinusoidal interference was merely counted from the image. The
image was 450 pixels high. The spatial model was thus constructed as follows:
\begin{align}
    \mathrm{filter} = A_2 + A_1\sin(x),
\end{align}
where
\begin{align}
    x = \texttt{linspace(0 + $\delta_1$, $15\times2\pi + \delta_2$)}
\end{align}
The parameter choice and meaning and final values are explained in table~\ref{tab:var}.
\begin{table}[h]
    \centering
    \caption{Parameters of the spatial model}
    \label{tab:var}
    \begin{tabular}{rlr}
        \toprule
        parameter  & value& meaning\\
        \midrule
        $A_1$ & 0.0303 & Amplitude of the sinusoidal noise\\
        $A_2$ & 0.1919& Background noise level \\
   $\delta_1$ & 0& Start phase (angular value) of the noise \\
   $\delta_2$ & 6.2929 & End phase of the noise \\
        \bottomrule
    \end{tabular}
\end{table}

The parameters seemed to be sufficient to capture the variability of the
spatial features of the noise. They were optimized by finding a value that
minimized MSE for each of the parameters, and then starting the cycle over
using the optimized values for other parameters until convergence was achieved.
For more detail, refer to the included source code.
\item[FFT]
\end{description}

The filtering using frequency domain was done by calculating the 2-dimensional
FFT of the image, finding the spikes corresponding to the frequency of the
interference pattern and removing them by filtering the spikes (only the spikes
were removed, though we also tested notch filtering). The FFT and the removal
of the spikes are shown in figures~\ref{fig:fft1} and \ref{fig:fft2}
respectively. At least when zooming in, you can see the high-contrast spikes in
the centerline that were removed. The image can be obtained using inverse
fourier transform.

\begin{figure}
        \centering
        \begin{subfigure}[b]{0.5\textwidth}
                \includegraphics[width=\textwidth]{fft.png}
                \caption{FFT of the image.}
                \label{fig:fft1}
        \end{subfigure}%
        ~ %add desired spacing between images, e. g. ~, \quad, \qquad etc.
          %(or a blank line to force the subfigure onto a new line)
        \begin{subfigure}[b]{0.5\textwidth}
                \includegraphics[width=\textwidth]{fft_removed.png}
                \caption{FFT with the spiked removed}
                \label{fig:fft2}
        \end{subfigure}
        \caption{FFT filtering}\label{fig:animals1}
\end{figure}

\subsubsection{Results}

The mean squared errors for the original noisy image and the filtering results
are shown in table~\ref{tab:mse}. The filtering
results and the error images are shown in figure~\ref{fig:animals3}.

Looking at the MSEs and the visual results, the spatial model works slightly
better as it removes all of the noisy waves from the image: There's still
something left in the frequency filtered image. Larger contribution to the
difference in the MSEs comes probably from the difference in overall brightness
and contrast in the image, which suffered in both filtering methods.

Overall, I'm proud of the performance of the ad hoc spatial filter and
parameter optimization implemented here.

\begin{table}[h]
    \centering
    \caption{MSEs of the images}
    \label{tab:mse}
    \begin{tabular}{rl}
        \toprule
        image & MSE\\
        \midrule
        noisy &  0.0294\\
    spatial filter&   0.0103\\
      frequency filter & 0.0208\\
        \bottomrule
    \end{tabular}
\end{table}

\begin{figure}[H]
        \centering
        \begin{subfigure}[b]{0.3\textwidth}
                 %\includegraphics[width=\textwidth]{topleft1.png}
                \caption{median filter}
        \end{subfigure}%
        ~ %add desired spacing between images, e. g. ~, \quad, \qquad etc.
          %(or a blank line to force the subfigure onto a new line)
        \begin{subfigure}[b]{0.3\textwidth}
                 %\includegraphics[width=\textwidth]{topright1.png}
                \caption{median filter}
        \end{subfigure}
\\
        \begin{subfigure}[b]{0.3\textwidth}
                 \includegraphics[width=\textwidth]{spatial_filtered.png}
                \caption{spatial model}
        \end{subfigure}%
        ~ %add desired spacing between images, e. g. ~, \quad, \qquad etc.
          %(or a blank line to force the subfigure onto a new line)
        \begin{subfigure}[b]{0.3\textwidth}
                 \includegraphics[width=\textwidth]{spatial_diff.png}
                \caption{spatial model}
        \end{subfigure}

        \begin{subfigure}[b]{0.3\textwidth}
                 \includegraphics[width=\textwidth]{fft_filtered.png}
                \caption{frequency domain filtering}
        \end{subfigure}%
        ~ %add desired spacing between images, e. g. ~, \quad, \qquad etc.
          %(or a blank line to force the subfigure onto a new line)
        \begin{subfigure}[b]{0.3\textwidth}
                 \includegraphics[width=\textwidth]{fft_diff.png}
                \caption{frequency domain filtering}
        \end{subfigure}
        \caption{Results of the different filters}\label{fig:animals3}
\end{figure}
}
\appendix
\section{Code}
\matlabscript{assignment2}{}
\end{spacing}

%\begin{thebibliography}{9}
%\bibitem{hc}
   %NCBI Map Viewer: Homo Sapiens(human),
   %\emph{Chromosome 22}.\\
   %http://www.ncbi.nlm.nih.gov/projects/mapview/maps.cgi?taxid=9607\&chr=22\\
   %retrieved Sept. 18 2012.

%\end{thebibliography}
\end{document}

%%%%%%%%%%%%%%%%%%%%%%%%%%%%%%%%%%%%%%%%%%%%%%%%%%%%%%%%%%%%%

%----------------------------------------------------------------------%
% The following is copyright and licensing information for
% redistribution of this LaTeX source code; it also includes a liability
% statement. If this source code is not being redistributed to others,
% it may be omitted. It has no effect on the function of the above code.
%----------------------------------------------------------------------%
% Copyright (c) 2007, 2008, 2009, 2010, 2011 by Theodore P. Pavlic
%
% Unless otherwise expressly stated, this work is licensed under the
% Creative Commons Attribution-Noncommercial 3.0 United States License. To
% view a copy of this license, visit
% http://creativecommons.org/licenses/by-nc/3.0/us/ or send a letter to
% Creative Commons, 171 Second Street, Suite 300, San Francisco,
% California, 94105, USA.
%
% THE SOFTWARE IS PROVIDED "AS IS", WITHOUT WARRANTY OF ANY KIND, EXPRESS
% OR IMPLIED, INCLUDING BUT NOT LIMITED TO THE WARRANTIES OF
% MERCHANTABILITY, FITNESS FOR A PARTICULAR PURPOSE AND NONINFRINGEMENT.
% IN NO EVENT SHALL THE AUTHORS OR COPYRIGHT HOLDERS BE LIABLE FOR ANY
% CLAIM, DAMAGES OR OTHER LIABILITY, WHETHER IN AN ACTION OF CONTRACT,
% TORT OR OTHERWISE, ARISING FROM, OUT OF OR IN CONNECTION WITH THE
% SOFTWARE OR THE USE OR OTHER DEALINGS IN THE SOFTWARE.
%----------------------------------------------------------------------%
